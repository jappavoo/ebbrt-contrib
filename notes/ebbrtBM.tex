\documentclass[11pt]{report}

\title{EbbRT BM Exploration}
\begin{document}
\maketitle

\chapter{Getting Started}

This is what I needed to do to get going with writing and testing
some Ebbs on our bare-metal environment.  The goal is to recreate
some of the "standard" Clustered Object benchmarks.

\section{Environment}
\begin{description}
\item{HOST:} kd-sesa.bu.edu 
\begin{verbatim}
Linux kd-sesa 3.13.0-1-generic #16-Ubuntu SMP 
Tue Jan 7 19:44:06 UTC 2014 x86_64 x86_64 x86_64 GNU/Linux
\end{verbatim}
\item{REPO:} 
\begin{verbatim}
git clone --recursive git@github.com:SESA/EbbRT-baremetal.git
\end{verbatim}
\end{description}

To get going once a clone has been done you need to cd into a particular build
directory and run make. eg.
\begin{verbatim}
cd EbbRT-baremetal/
jappavoo@kd-sesa:~/Work/EbbRT-baremetal$ ls
build  build.mk  ext  misc  sys
jappavoo@kd-sesa:~/Work/EbbRT-baremetal$ cd build/
jappavoo@kd-sesa:~/Work/EbbRT-baremetal/build$ ls
debug  opt
jappavoo@kd-sesa:~/Work/EbbRT-baremetal/build$ cd ..
jappavoo@kd-sesa:~/Work/EbbRT-baremetal$ ls
build  build.mk  ext  misc  sys
jappavoo@kd-sesa:~/Work/EbbRT-baremetal$ cd build
jappavoo@kd-sesa:~/Work/EbbRT-baremetal/build$ ls
debug  opt
jappavoo@kd-sesa:~/Work/EbbRT-baremetal/build$ cd opt
jappavoo@kd-sesa:~/Work/EbbRT-baremetal/build/opt$ ls
Makefile
jappavoo@kd-sesa:~/Work/EbbRT-baremetal/build/opt$ make
  CXX sys/acpi.o
  CXX sys/apic.o
  AS sys/boot.o
  ...
    CXX sys/gthread.o
../../ext/toolchain/bin/x86_64-pc-ebbrt-g++ -std=gnu++11 -U ebbrt -MD 
-MT sys/runtime.o -MP -g3 -O4 -flto -DNDEBUG -Wall -Werror 
-fno-stack-protector -I ../.. -I ../../ext/acpica/include 
-I ../../ext/boost/include -I ../../ext/lwip/include 
-I ../../ext/tbb/include -iquote ../../sys -iquote 
../../ext/lwip/include/ipv4/  -c -o sys/runtime.o 
  ../../sys/runtime.cc
  LD ebbrt.elf
  STRIP ebbrt.elf.stripped
  MKRESCUE ebbrt.iso
Enabling BIOS support ...
xorriso 1.3.2 : RockRidge filesystem manipulator, libburnia project.
\end{verbatim}

To see all the gory details you can run the make system in verbose mode eg.
{\tt make V=1}.

This should leave you with a basic BM bootable ISO image:
\begin{verbatim}
jappavoo@kd-sesa:~/Work/EbbRT-baremetal/build/opt$ ls 
ebbrt.elf  ebbrt.elf.stripped  ebbrt.iso  ext  Makefile  sys
\end{verbatim}

\newpage
You should be able to run this on kvm via its qemu interface (Note: to run via kvm you will need to be in the kvm group).
\begin{verbatim}
jappavoo@kd-sesa:~/Work/EbbRT-baremetal/build/opt$ qemu-system-x86_64 \
-m 2G \
-numa node,cpus=0,mem=1G \
-numa node,cpus=1,mem=1G \
-smp sockets=2 \
-cpu host \
-gdb tcp::1235 \
-enable-kvm \
-nographic \
-net nic,model=virtio \
-net vde,sock=/tmp/vde_switch0 \
-cdrom ebbrt.iso
\end{verbatim}

Currently by default this will get you to the point
that the system will be waiting for a dhcp answer which will never come.  

{\tiny
\begin{verbatim}
Warning: vlan 0 is not connected to host network
EbbRT Copyright 2013-2014 SESA Developers
e820_map:
000000000000000000-0x000000000009fbff usable
0x000000000009fc00-0x000000000009ffff reserved
0x00000000000f0000-0x00000000000fffff reserved
0x0000000000100000-0x000000007fffdfff usable
0x000000007fffe000-0x000000007fffffff reserved
0x00000000feffc000-0x00000000feffffff reserved
0x00000000fffc0000-0x00000000ffffffff reserved
ACPI: RSDP 0xf1650 000014 (v00 BOCHS )
ACPI: RSDT 0x7fffe300 000038 (v01 BOCHS  BXPCRSDT 00000001 BXPC 00000001)
ACPI: FACP 0x7fffff80 000074 (v01 BOCHS  BXPCFACP 00000001 BXPC 00000001)
ACPI: DSDT 0x7fffe340 001137 (v01   BXPC   BXDSDT 00000001 INTL 20100528)
ACPI: FACS 0x7fffff40 000040
ACPI: SSDT 0x7ffff6a0 000899 (v01 BOCHS  BXPCSSDT 00000001 BXPC 00000001)
ACPI: APIC 0x7ffff5b0 000080 (v01 BOCHS  BXPCAPIC 00000001 BXPC 00000001)
ACPI: HPET 0x7ffff570 000038 (v01 BOCHS  BXPCHPET 00000001 BXPC 00000001)
ACPI: SRAT 0x7ffff480 0000F0 (v01 BOCHS  BXPCSRAT 00000001 BXPC 00000001)
Local APIC: ACPI ID: 0 APIC ID: 0
Local APIC: ACPI ID: 1 APIC ID: 1
IO APIC: ID: 0
Interrupt Override: 0 -> 2
Interrupt Override: 5 -> 5
Interrupt Override: 9 -> 9
Interrupt Override: 10 -> 10
Interrupt Override: 11 -> 11
NMI on all processors: LINT1
SRAT CPU affinity: 0 -> 0
SRAT CPU affinity: 1 -> 1
SRAT Memory affinity: 000000000000000000-0x000000000009ffff -> 0
SRAT Memory affinity: 0x0000000000100000-0x000000003fffffff -> 0
SRAT Memory affinity: 0x0000000040000000-0x000000007fffffff -> 1
Early Page Allocator NUMA mapping 0x0000000000782000-0x000000003fff9fff -> 0
Early Page Allocator NUMA mapping 0x0000000040000000-0x000000007fffdfff -> 1
Core 1 online
PCI: 0:1:1 - BAR4 0x000000000000c020 - 0x000000000000c030
PCI: 0:2:0 - BAR0 0x00000000fc000000 - 0x00000000fe000000
PCI: 0:2:0 - BAR1 0x00000000febf1000 - 0x00000000febf2000
PCI: 0:3:0 - BAR0 0x000000000000c000 - 0x000000000000c020
PCI: 0:3:0 - BAR1 0x00000000febf0000 - 0x00000000febf1000
0:3:0
MSIX - 3 entries at BAR1:0
Mac Address: 52:54:00:12:34:56
\end{verbatim}
}

At this point you are looking at the output from QEMU on its simulated serial line.  You can escape to the QEMU monitor via "CTRL-A c" or quit directly via "CTRL-A x".  

You can also attach gdb to the qemu debug stub that was configured via the qemu command line parameters.
Eg.
\begin{verbatim}
jappavoo@kd-sesa:~/Work/EbbRT-baremetal/build/opt$ gdb ebbrt.elf
(gdb) target remote localhost:1235
Remote debugging using localhost:1235
kabort (args#0=...) at ../../sys/debug.hpp:13
13        kprintf(args...);
(gdb) 
\end{verbatim}

\chapter{Hacking}

First order of business is to get a simple baremetal
up that gets all the way to issuing a application
startup message and then sits in the event loop.
To do this must first turn off the networking stuff
and get ourselves up to our own startup code.

\section{Disabling Networking}

Ok it looks like the place to start is {\tt sys/main.cpp:109}
We seem to be unconditionally starting the network. 
There obvious things to do:
\begin{enumerate}
\item passing argument to avoid network startup
\item make conditional on probe of nic being successful
\item refactor code so that startup can be more easily customized
\end{enumerate}
Kludging until I understand more and grok our config strategy.
So for the moment simply added \#ifdef NETWORKING around network bring-up.
and tested this seems to get me to an initialized system with out networking
as I desirer.   

Added  to by ebbrt-contrib a qbm script to launch a bm instance with
a set of default configurable arguments. eg:
{\tt NETWORKING="" ebbrt-contrib/script/qbm} is what I am using to test 
my baremetal app with no networking.

\section{Initial "App" Event -- Hello World}

First want to explore how to added an event that occurs after
the "standard" System initialization.  Which I assume ends
at the point of seeing the "System initialization complete" message.

First attempt was adding the following after the first SpawnLocal in main.cpp:
\begin{verbatim}
    event_manager->SpawnLocal([]() {
	kprintf("Hello World!!!\n");
      });
\end{verbatim}

But this interleaved the Hello World before the system initialization was complete:
\begin{verbatim}
...
Early Page Allocator NUMA mapping 0x0000000040000000-0x000000007fffdfff -> 1
Hello World!!!
Core 1 online
System initialization complete
\end{verbatim}

Need to understand why.
Next try is to put a spawn at the end of the initialization event.
Ok this seems to have the effect that I want but not sure this is really 
the "right" way to do it.  After I have a better understanding of how to 
think about Events and the Event loop I will document and redo as needed.

I would also like to understand what the correct way is for turning a hardware
interrupt into a invocation of a specific method on a specific instance of an Ebb is (I assume also on a new Event).

\subsection{hello app}
Ok next step is to get Hello World building as its own ISO and runs
on its own App specific initialization event that is spawned at the 
end of the standard initialization event (no consideration for 
multi-core issues at this point)

First approach I am trying is to create a new set of files in the same
directory and modify as little as possible in the current files.

\begin{enumerate}
\item modified main.cpp:
\begin{verbatim}
extern void appmain() __attribute__((weak));
...
    kprintf("System initialization complete\n");

    if (appmain) {
      event_manager->SpawnLocal(appmain);
    } else {
      kprintf("No appmain found...\n");
    }

  });
\end{verbatim}
\item added a hello.cxx
\begin{verbatim}
#include <sys/debug.hpp>

using namespace ebbrt;

void appmain()
{
  kprintf("Hello World!!!\n"); 
}
\end{verbatim}
\item added hello.mk
\begin{verbatim}
include ./Makefile

all: hello.iso

hello.o: hello.cpp

hello.iso: hello.elf.stripped
	$(call quiet, $(mkrescue), MKRESCUE $@)

hello.elf.stripped: hello.elf
	$(call quiet, $(strip), STRIP $@)

app_objects = hello.o

LDFLAGS := -Wl,-n,-z,max-page-size=0x1000 $(optflags)
hello.elf: $(app_objects) $(objects) sys/ebbrt.ld $(runtime_objects)
	$(call quiet, $(CXX) $(LDFLAGS) -o $@ $(app_objects) $(objects) \
		-T $(src)/sys/ebbrt.ld $(runtime_objects), LD $@)

clean::
	-$(RM) $(wildcard $(app_objects) hello.elf)
\end{verbatim}
\item small modification to build.mk...changed clean: to clean:: to allow extension
\end{enumerate}

So now I build this by running {\tt make -f hello.mk}. This produces hello.iso
which contains a defined appmain:
\begin{verbatim}
jappavoo@kd-sesa:~/Work/EbbRT-baremetal/build/opt$ nm hello.elf | grep appmain
0000000000118000 T _Z7appmainv
\end{verbatim}
However running make allow against the default Makefile creates the standard ebbrt.iso with appmain left as an undefined weak:
\begin{verbatim}
jappavoo@kd-sesa:~/Work/EbbRT-baremetal/build/opt$ nm ebbrt.elf | grep appmain
                 w _Z7appmainv
\end{verbatim}

Running hello.iso (eg. {\tt NETWORKING="" ~/Work/ebbrt-contrib/scripts/qbm hello.iso}) produces the desired effect:
\begin{verbatim}
...
Core 1 online
System initialization complete
Hello World!!!
\end{verbatim}

and running ebbrt.iso (eg. {\tt NETWORKING="" ~/Work/ebbrt-contrib/scripts/qbm})
produces:
\begin{verbatim}
Core 1 online
System initialization complete
No appmain found...
\end{verbatim}

\subsection{Hello App in its own standalone directory}

Did two attempts at this.  First was trivial approach but I was not really happy with the outcome.  So I did something that required a little more surgery on the current build.mk.
\begin{enumerate}
\item modified build.mk:
\begin{quotation}
Removed explicit targets for ebbrt.elf, ebbrt.elf.stripped and ebbrt.iso and added default rules for \%.elf, \%.elf.stripped, \%.iso.  I also removed default all target, added a .PRECIOUS directive to ensure all intermediates are preserved (this should also help
interactions with .depend files on initial build), added a reset of the .DEFAULT\_GOAL at the end of the file.  All these changes are to make build.mk more friendly to inclusion.   
\end{quotation}
\item modified opt/Makefile and debug/Makefile:
\begin{quotation}
Added following two lines at the bottom of both
\begin{verbatim}
.PHONY: all
all: ebbrt.iso
\end{verbatim}
\item Now my out of source application build infrastructure looks like this:
\begin{verbatim}
./apps:
hello
./apps/hello:
bm
./apps/hello/bm:
build  ebbrtbm.mk  src
./apps/hello/bm/build:
debug  Makefile  opt
./apps/hello/bm/build/debug:
Makefile
./apps/hello/bm/build/opt:
Makefile
./apps/hello/bm/src:
hello.cpp
\end{verbatim}
Where the apps directory is a sub directory of my ebbrt-contrib repo. Hello is a sub dir which splits for baremetal and then the debug levels.  src/hello.cpp is my common application source.  ebbrt.mk is a common make file that contains the setup to a ebbrt source tree and then the build/debug/Makefile and build/opt/Makefile define the build for the particular optimization flags.
\end{quotation}
\end{enumerate}
To build things one simple runs make in the appropriate subdir specifying the location of the ebbrtbm source eg:
{\tiny
\begin{verbatim}
~/Work/ebbrt-contrib/apps/hello/bm/build/opt$ EBBRT_SRCDIR=~/Work/EbbRT-baremetal make
\end{verbatim}
}
\newpage
The Makefiles are as follows:
\begin{verbatim}
bm/ebbrtbm.mk
ifeq ($(strip ${EBBRT_SRCDIR}),)
 $(error You must set EBBRT_SRCDIR to the path of an EbbRT source tree)
endif
src = ${EBBRT_SRCDIR}
VPATH = ${EBBRT_SRCDIR}:../../src

ifeq ($(strip ${EBBRT_OPTFLAGS}),)
 $(error You must set EBBRT_OPTFLAGS as you need. eg. -g3 -O4 -flto -DNDEBUG for an optimized build or -g3 -O0 for a non-optimized debug build)
endif

optflags = ${EBBRT_OPTFLAGS}
include ${EBBRT_SRCDIR}/build.mk
--------
opt/Makefile
EBBRT_OPTFLAGS=-g3 -O4 -flto -DNDEBUG
app_objects = hello.o

include ../../ebbrtbm.mk

.PHONY: all
all: hello.iso
----
debug/Makefile
EBBRT_OPTFLAGS=-g3 -O0
app_objects = hello.o

include ../../ebbrtbm.mk

.PHONY: all
all: hello.iso
\end{verbatim}

With this I get a simple model for an app which has its own source and build dirs
and easily produces a baremetal image.

\subsection{Hello App with and Ebb}







 
\end{document}
