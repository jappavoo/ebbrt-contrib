\documentclass[11pt]{article}
\title{Boot Sequence Wiki Comments}
\begin{document}
\maketitle
\begin{description}
    \item{Entry} Nice description!  Thank you
    \item{Main} cool ... does disabling the legacy pic also imply ensuring all 
                sources of interrupts are squashed... eg I assume the APIC is
                not automatically enabled in an unknown state.
    \item{Memory Information Discovery and Page Mappings} So this a singleton run on the boot processor.  Do we have a sense of how long this can take on a machine that has an ugly layout?
    Will we need something that turns some of this off if we don't really need more than 
    the first Gig for an app that wants to use a dynamic node for very fast event processing?
    Eg.  can we configure a static memory system so that we optimize when we know the machine
     details and discovery is not an issue.  Eg come up in a state that the Memory allocator
     representatives are backed in? 
    \item{Ebb Initialization and Memory Allocators}
      \begin{enumerate}
        \item given that the boot processor has touch all of memory is there utility 
              in leaving any extra info from this walk in the per-numa regions?
        \item should the boot processor flush it's caches to alleviate cache traffic as the other cores will touch the memory?
        \item Assume this sequence is still running on a single boot core
              at first but will eventually be called on all cores
        \item assume the page allocator, since it is shared across a NUMA node,
              is lock based?  Is there any consideration of cache coherency traffic 
              and critical section lengths within a NUMA node for the data-structures?
              I assume not as I am guessing the expectation that is a hierarchy and the       
              lower levels will address these issues and the frequency of access to 
              this is lower and its costs amortized by the below layers which can
              be tuned for a particular use case.
        \item slab is a factory to create specific instances tuned to particular parameters  (nice).
        \item Do you have in your mind a case/workload that you could show that the gp 
              parameterization is not what you want.  
        \item LocalIdmap:  My only question is why did the other stuff that happened
              before not have a dependency on LocalIdMap.  Is there implicitly a special
              class of "Static"/"Base"/"Initialization" Ebbs that is distinct from 
              the ones that get created post LocalIdMap initialization.
        \item is EbbAllocator just and interface around the LocalIdMap?  If so 
              is there any sense in encapsulating it?
        \end{enumerate}
    \item{Starting the First Event}    
      \begin{enumerate}
        \item Is VMemAllocator init per core or is this shared?  Is this an Ebb?
        \item EventManager: producer of stacks
        \item SpawnLocal:  where is this queue?  Who is responsible for the resources and ordering on the queue?
        \item StartProcessingEvents : is this the Eventloop?  Is it a loop or is it 
              a function called by an external agent?
    \end{enumerate}
    \item{First Event}
\begin{enumerate}
  \item do I think of this as the crt0 code?
  \item apic so now we are back to system code opposed to language initialization?
        At this point could interrupt driven events occur?
  \item Timer:  Is there an inherent timer event that is running too?
  \item Ok so everyone initialize with a variant of the same code to get to 
        all the first events terminating in a synchronous manner.  Is there 
        a flag that turns off amp bring up or limits the number of cores?
        Is there any utility is offering the ability to allow ragged starts?
        Eg.  app code might be smart enough to get going and only use services
        when they are ready?
\end{enumerate}
\end{description}
VERY NICE!  Random though do we want a basic notification service as a programming paradigm to facilitate rendezvous style early synchronization. Eg.  When this becomes available do this.  I know that futures capture this but I just need to think more about
more async continuation scenarios.  
\end{document}
